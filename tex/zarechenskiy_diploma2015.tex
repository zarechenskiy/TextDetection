\documentclass{matmex-diploma-custom}
\begin{document}
\filltitle{ru}{
    chair              = {Кафедра информационно-аналитических систем},
    title              = {Обнаружение текста в коллекциях изображений, содержащих текст на разных языках},
    type               = {bachelor},
    position           = {студента},
    group              = 441,
    author             = {Зареченский Михаил Алексеевич},
    supervisorPosition = {к.\,ф.-м.\,н.},
    supervisor         = {Васильева Н.\,С.},
    reviewerPosition   = {ст. преп.},
    reviewer           = {Рецензент А.\,И.},
    chairHeadPosition  = {д.\,ф.-м.\,н., профессор},
    chairHead          = {Новиков Б.\,А.},
%   university         = {Санкт-Петербургский Государственный Университет},
%   faculty            = {Математико-механический факультет},
%   city               = {Санкт-Петербург},
%   year               = {2013}
}
\filltitle{en}{
    chair              = {Chair of The Analytical Information Systems},
    title              = {Text detection in image collections with multilingual text},
    author             = {Mikhail Zarechenskiy},
    supervisorPosition = {},
    supervisor         = {Natalia Vassilieva},
    reviewerPosition   = {assistant},
    reviewer           = {assistant},
    chairHeadPosition  = {professor},
    chairHead          = {Boris Novikov},
}
\maketitle
\tableofcontents
\setcounter{tocdepth}{3}
% У введения нет номера главы
\section*{Введение}
Последние достижения в области цифровых технологий позволяют делать снимки практически с любого мобильного устройства. Как результат, количество фотографий, сделанных пользователями, растёт каждый день. В тоже время, коллекции с большим количеством изображений очень долго аннотировать вручную и поэтому они часто остаются без каких-либо сведений, кроме аннотаций, сделанных самим устройством.\\
\indent Наличие текста на изображении часто даёт важную информацию о семантике изображения, как, например, название магазина, улицы, название предмета на изображении и так далее. Проаннотированные изображения можно использовать для различных задач анализа изображений, таких как поиск изображений, автоматическая навигация. Другой популярной идеей является моментальный перевод текста на изображении с мобильного устройства. Снятая фотография, содержащая иностранный язык, обрабатывается на предмет обнаружения и распознавания текста для дальнейшего перевода на требуемый язык.\\
\indent Такие сценарии использования говорят о том, что изображения часто могут быть получены с мобильного устройства и иметь низкое разрешение, размытие, шум. Более того, часто заранее не может быть известен язык, на котором написан текст на изображении.\\ 
\indent Актуальной исследовательской задачей является эффективное обнаружение и распознавание текста на изображениях. Обнаружение текста является одним из самых важных шагов при дальнейшем распознавании текста. В данной работе будет исследоваться именно задача обнаружения текста.\\
\indent Важной особенностью данной работы является рассмотрение задачи обнаружения текста по отношению к изображениям, содержащим текст на разных языках. Будет показано, что эффективность многих существующих подходов к обнаружению текста может сильно зависеть от языка, представленного на изображении. Для того, чтобы оценить инвариантоность тех или иных методов к языку, будет рассмотрена теоретическая основа методов, а также будут поставлены эксперименты с различными данными, как содержащими только текст на английском языке, так и с данными, содержащими текст на разных языках.
\section*{Постановка задачи}
В рамках дипломной работы были поставлены следующие задачи:
\begin{itemize}
\item Проанализировать наиболее эффективные методы обнаружения текста
\item Описать и выделить основные этапы работы алгоритмов предментной области
\item Аналитически определить зависимость каждого из этапов алгоритмов обнаружения текста к языку
\item Поставить ряд экспериментов на данных, как содержащих только английский текст, так и на данных, содержащих текст на разных языках
\item Провести эмпирический анализ и сделать выводы о зависимости алгоритмов предментной области к языку
\end{itemize}
\section{Обзор существующих подходов}
Для наиболее эффективного распознавания текста, первым шагом текст на изображении должен быть точно обнаружен, однако это является довольно сложной задачей в связи с большой вариацией представления текста на  изображении. Текст может иметь различные вариации шрифта, стиля, размера, искажения, иметь различную контрастность из-за разных условий освещения. Всё изображение так же может сильно варьироваться, следует учитывать низкое разрешение, низкую контрастность, неоднородный фон. Такое разнообразие порождает различные подходы к обнаружению текста.\\
\indent С одной стороны, априорное знание о структуре текста, привело к развитию алгоритмов, которые можно поделить на три категории: алгоритмы, основанные на анализе текстуры, алгоритмы на основе компонент связности~\cite{Yin01},~\cite{Gomez02},~\cite{Yalniz04},~\cite{Chen05} и гибридные методы~\cite{Barinova03}.\\
\indent С другой стороны, методы, основанные на глубоком обучении (deep learning), использующие небольшое количество априорной информации о данных, получили широкое распространение в таких областях задач анализа изображений, как классификация и распознавание объектов. В том числе, существуют и  алгоритмы обнаружения текста, работающие на основе свёрточных нейронных сетей~\cite{wang},~\cite{coates}.\\
\indent Далее рассмотрим подробнее алгоритмы обнаружения текста из каждого класса.  
\subsection{Методы, использующие априорное знание о структуре текста}
\indent К алгоритмам, использующим априорное знание о структуре текста относят алгоритмы, основанные на анализе текстуры, алгоритмы на основе компонент связности и гибридные методы.\\
\indent Алгоритмы, основанные на анализе текстуры, выделяют текстурные особенности из изображения, после чего классификатор идентифицирует наличие текста. Как правило, текстурные особенности выделяются с использованием техники скользящего окна. Данная процедура довольно трудоёмка, и, более того, необходимо провести данную операцию для разных масштабов. Алгоритмы, основанные на данном подходе имеют тенденцию к низкой производительности и значительно теряют точность при падении качества изображений.\\
\indent В тоже время алгоритмы, основанные на связных компонентах, извлекают из изображений отдельные регионы - кандидаты в символы. Далее применяются различные эвристики для фильтрации выделенных компонент. Оставшиеся компоненты(регионы) группируются в текст. Компоненты группируются в текст как правило двумя способами: на основе геометрических особенностей и при помощи методов кластеризации. На последнем шаге также возможна дополнительная фильтрация для устранения ошибок второго рода.\\
\indent Гибридные методы на первом шаге производят сегментацию изображения таким образом, чтобы символы, расположенные на изображении, имели отчётливую границу с фоном. Далее выделяются компоненты связности и применяются различные эвристики для исключения компонент, не представляющих символы. На последнем шаге компоненты группируются в текст.\\
\indent Стоит отметить, что в соответствии с результатами соревнования "Multi-script Robust Reading Competition"~\cite{icdar2013} на конференции ICDAR 2013, подход, основанный на связных компонентах, показал себя, как наиболее эффективный и производительный по сравнению с остальными подходами. Поэтому, далее данный подход будет рассматриваться более подробно.
\subsection{Методы, основанные на глубоком обучении}
\indent Методы, основанные на глубоком обучении, позволяют отойти от выбора вручную признаков для обнаружения текста. Признаки извлекаются из коллекции изображений, предназначенной для обучения. При этом вид признаков зависит от метода обучения. Например, при использовании свёрточных нейронных сетей~\cite{lecun}, признаки ограничиваются свёрточными масками и их комбинациями.\\
\section*{Заключение}

\bibliographystyle{ugost2008ls}
\bibliography{diploma}
\end{document}
